\documentclass[11pt, a4paper]{article}

\usepackage{fontspec}
\usepackage{geometry}
\geometry{a4paper, margin=2cm}
\usepackage{titlesec}
\usepackage{titling}
\usepackage{enumitem}
\setlist{nolistsep}
\usepackage{hyperref}
\usepackage{kotex}
\usepackage{array}
\usepackage{setspace}
\usepackage{xcolor}
\usepackage{enumitem}
\usepackage{fontawesome}
\definecolor{pastelgreen}{RGB}{120, 220, 120}
\definecolor{darkergreen}{RGB}{0, 128, 0}
\definecolor{darkeryellow}{RGB}{254, 176, 36}
\newcommand{\textbr}[1]{\textbf{\textcolor{bonusSteelBlue}{#1}}}
\newcommand{\textcb}[1]{\textcolor{bonusSteelBlue}{(#1)}}
\definecolor{bonusTeal}{RGB}{0, 128, 128}
\definecolor{bonusCoral}{RGB}{255, 127, 80}
\definecolor{bonusPlum}{RGB}{142, 69, 133}
\definecolor{bonusOlive}{RGB}{107, 142, 35}
\definecolor{bonusSteelBlue}{RGB}{70, 130, 180}
\definecolor{mediumGray}{RGB}{246,246,246}
\hypersetup{
    colorlinks=true,
    linkcolor=blue,
    filecolor=magenta,
    urlcolor=blue,
    pdftitle={Overleaf Example},
    pdfpagemode=FullScreen,
    }


\titleformat{\section}
{\large\bfseries}
{}
{0em}
{}[\titlerule]

\titleformat{\subsection}[runin]
{\bfseries}
{}
{0em}
{}

\renewcommand{\maketitle}{
\begin{center}
{\huge\bfseries
\theauthor}

\vspace{.25em}
010-2805-0547 | tyeirdoo06@gmail.com
\end{center}
}

\begin{document}
\pagecolor{mediumGray}
\onehalfspacing

\title{이력서}
\author{두혁진}

\maketitle

\section{개발 교육}
\textbf{42 서울}(2021년 9월 - 현재)\\
개요: 동료학습, PBL 기반 혁신 교육기관

% 학습 내용: C ~\cdot~ C++ ~\cdot~ Linux ~\cdot~ Shell ~\cdot~ Web

\section{보유 기술 \textcolor{bonusSteelBlue}{(관심분야)}}
\begin{tabular}{r| p{0.7\textwidth} c}
    \textbf{Programming Language~\cdot~}& \textbf{C / C++}~\cdot~Bash~\cdot~TypeScript~\cdot~HTML5~\cdot~CSS3\\
    \textbf{Methology}& \textcb{\textbf{Rust}, TDD, Fucntional Programming}\\
    \textbf{Framework~\cdot~Library}& Next.js~\cdot~React~\cdot~TailwindCSS \newline \textcb{Vulkan, Three.js}\\
    \textbf{Server~\cdot~DB}& Mariadb, Nginx, Lighttpd \newline \textcb{GraphQL}\\
    \textbf{Tooling~\cdot~DevOps}& \textbf{Make}, Docker compose, Github, ansible, Prometheus, Grafana \newline \textcb{Open Stack}\\
    \textbf{Environment}& Linux(debian, alpine), MacOS, GCP \newline \textcb{AWS}\\
    \textbf{Editor}& Vim~\cdot~Neovim \newline \textcb{Zed}
\end{tabular}

\section{프로젝트}
\textbf{minitalk (C)}

목표: 시그널(USR1, USR2)을 이용하여 데이터 통신 구현, 유니코드 호환

추가 구현: \textbr{해밍 코드를 이용한 에러 정정}\\
\textbf{minishell (C)}

목표: Bash의 기능을 일부 구현

추가 구현: \textbr{LALR parser 구현}\\
\textbf{inception (docker)}

목표: Docker compose를 사용하여 LEMP환경 설정(모든 Image 직접 작성)

추가구현: \textbr{Prometheus와 Grafana를 사용하여 MySQL + Radis 모니터링 환경 구축}\\
\textbf{ft\_irc (C++98)}

목표: 멀티 클라이언트 지원 IRC 서버 구현, 멀티플렉싱과 논블록킹 I/O

추가구현: \textbr{멀티스레딩, 스마트포인터사용}\\

\section{개선 / 문제해결 사례}
\textbf{Humans of 42}(2023년 10월 24일 - 11월 24일)

개요: 인터뷰 사진과 글을 제공하는 사이트를 Next.js와 github.io를 이용한 스태틱 페이지로 재구성.

소개글: \textbf{\textit{\href{https://42humans.com}{Humans of 42}는 42서울 사람들의 시시콜콜한 이야기를 담습니다.}}\\\\

\begin{tabular}{l l}
    1. 문제:&사이트 인증서 만료 지속 발생\\
    2. 원인:&유지보수자 부재\\
    3. 해결:&오래된 기술스택을 next.js로 일신, static page로 변경\rightarrow github.io를 이용하여 배포\\
    4. 평가:&Github page에서 인증서를 자동 갱신\\
    5. 비고:&static page로 바꾸면서 서버 비용 기존 월 5만원 수준에서 0원으로 절감. \\
            &원본 이미지를 직접 제공, 이미지 퀄리티 상승.
\end{tabular}

\section{도전 / 실패 사례}
\textbf{42 Region}(2022년 9월 - 2022년 12월)

개요: 42 Seoul 내부에서 활용할 수 있는 프라이빗 클라우드 서비스 구축 (OpenStack 활용)\\

\begin{tabular}{l l}
    1. 문제:&인스턴스 생성 후 외부 인터넷 연결 불가\\
    2. 원인:&42 Seoul 시설망에서 switch의 tagged VLAN 설정 미비\\
    4. 평가:&L1\~L2레벨 지식 부족\\
    3. 대처:&물리계층부터 네트워크 설계 및 지식을 보충하기 위해 팀원들로 스터디 구성\\
            &설계에 필요한 최소한의 지식의 지도 형성\\
\end{tabular}


\section{학력}
\textbf{금강대학교 일본어 통번역학과} | 학사\\
2014년 3월 입학\\
\hspace*{0.5cm}└─\textbf{히토츠바시 대학 경제학과 (일본 국비 장학생)}\\
\hspace*{1cm}2015년 9월 - 2016년 8월 (2학기)\\
\hspace*{0.5cm}└─\textbf{레이타쿠 대학 경제학과}\\
\hspace*{1cm}2017년 4월 - 2017년 8월 (1학기)\\
2019년 2월 졸업\\

\section{병역 사항}
\textbf{대한민국 육군} | 병장 만기전역\\
역할: 작전병(행정)
복무 기간: 2019년 7월 - 2021년 2월

\section{경력}
\textbf{Watcha (계약직)}

업무: 콘텐츠 데이터 보충 및 관리, 번역(비계발)

근무 기간: 2019년 4월 - 2019년 7월, 2021년 3월 - 2021년 7월\\
\textbf{42 서울 운영팀 (인턴)}

업무: 42서울 내부 과제 명세 작성, ansible을 이용한 컴퓨터 클러스터 관리

근무 기간: 2023년 10월 - 2023년 12월

\section{수상 이력}
\textbf{정보통신기획평가원 원장상}\\
\hspace*{0.5cm}행사명: 2023 INNO-CON (이노베이션 아카데미 성과 공유 컨퍼런스)\\
\hspace*{0.5cm}수상 사유:\\
\hspace*{1cm}역할: 42 Seoul 코알리숑 마스터(학생 대표)\\
\hspace*{1cm}기여분야:
\begin{itemize}
    \begin{itemize}
        \item 커뮤니티 이벤트 개최
        \item 시설 환경 개선
    \end{itemize}
\end{itemize}
\hspace*{0.5cm}성과: 커뮤니티 활성화 및 학생, 운영진 간 소통 향상

\end{document}
